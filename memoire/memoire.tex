% ######################################################################
%
% 			PREAMBULE
%
% ######################################################################

% -------------------------------------------------------
% Ce document est un raport. cinquantaine de pages / plusieurs sections
% -------------------------------------------------------
\documentclass[a4paper,12pt]{report}

% -------------------------------------------------------
% Utiliser latin1 pour les accents
% ne pas utiliser utf8 et Latex n'aime pas utf8 ET latin1
%\usepackage[utf8]{inputenc}
% -------------------------------------------------------
\usepackage[utf8]{inputenc}

% utiliser plutôt french, frenchb est obsolete
\usepackage[T1]{fontenc}
\usepackage[english,french]{babel}

\usepackage{amsmath}
\usepackage{amssymb}
\usepackage{amsfonts}
\usepackage{array}
\usepackage{booktabs}
\usepackage{diagbox} % barre oblique pour les cellule à 2 entrées
\usepackage[pdftex]{graphicx}
\usepackage{hhline}
\usepackage{hyperref}
\usepackage{lmodern}
\usepackage{multicol}
\usepackage{subcaption}
\usepackage{supertabular}
\usepackage{textcomp}
\usepackage{xcolor}
\usepackage{xspace}

% -------------------------------------------------------
% Théorèmes
% -------------------------------------------------------
\usepackage{amsthm}
\theoremstyle{plain}				% Choix du style
\newtheorem{theoreme}{Théorème}	% Définition de l'environnement 1
\newtheorem{example}{Exemple}
\theoremstyle{definition}				% Choix du style
\newtheorem{definition}{Définition} %[section]	% Définition de l'environnement définition

% -------------------------------------------------------
% Pour les ALGORITHMES
% linesnumbered	: les lignes sont numérotées
% ruled			: Le caption est en haut et bordé de lignes horizontale
% vlined		: Regroupement des bloc d'instructions par ligne verticales
% -------------------------------------------------------
\usepackage[linesnumbered, ruled, vlined, french]{algorithm2e}

% Commandes en français:
\SetKwInput{KwRes}{R\'esultat}%
\SetKw{WEntree}{\textcolor{red}{Entrée}}
\SetKw{WEntrees}{\textcolor{red}{Entrées}}
\SetKw{WSaisir}{Saisir}
\SetKw{WInitialisation}{\textcolor{red}{Initialisation}}
\SetKw{WTraitement}{\textcolor{red}{Traitement}}
\SetKw{WAssigne}{\textcolor{blue}{prend la valeur}}
\SetKw{WSortie}{\textcolor{red}{Sortie}}
\SetKw{WSorties}{\textcolor{red}{Sorties}}
%
\SetKwIF{Si}{SinonSi}{Sinon}{Si}{alors}{sinon si}{sinon}{fin si}
\SetKwFor{Tq}{Tant que}{faire}{fin tantque}
\SetKwFor{Pour}{Pour}{faire}{fin pour}
\SetKw{WAfficher}{Afficher}
\SetKwRepeat{Repeter}{répéter}{jusqu'à}%
%
\SetKw{Return}{\textcolor{red}{Renvoyer}}%
%
\SetKwProg{Init}{init}{}{}
% mettre les commentaire des algos en bleu

% -------------------------------------------------------
% Pour les ALGORITHMES façon PYTHON
% -------------------------------------------------------
\usepackage{listings}
\lstset{literate={á}{{\'a}}1 {é}{{\'e}}1 {í}{{\'i}}1 {ó}{{\'o}}1 {ú}{{\'u}}1{Á}{{\'A}}1 {É}{{\'E}}1 {Í}{{\'I}}1 {Ó}{{\'O}}1 {Ú}{{\'U}}1{à}{{\`a}}1 {è}{{\`e}}1 {ì}{{\`i}}1 {ò}{{\`o}}1 {ù}{{\`u}}1{À}{{\`A}}1 {È}{{\'E}}1 {Ì}{{\`I}}1 {Ò}{{\`O}}1 {Ù}{{\`U}}1{ä}{{\"a}}1 {ë}{{\"e}}1 {ï}{{\"i}}1 {ö}{{\"o}}1 {ü}{{\"u}}1{Ä}{{\"A}}1 {Ë}{{\"E}}1 {Ï}{{\"I}}1 {Ö}{{\"O}}1 {Ü}{{\"U}}1{â}{{\^a}}1 {ê}{{\^e}}1 {î}{{\^i}}1 {ô}{{\^o}}1 {û}{{\^u}}1{Â}{{\^A}}1 {Ê}{{\^E}}1 {Î}{{\^I}}1 {Ô}{{\^O}}1 {Û}{{\^U}}1{œ}{{\oe}}1 {Œ}{{\OE}}1 {æ}{{\ae}}1 {Æ}{{\AE}}1 {ß}{{\ss}}1{ű}{{\H{u}}}1 {Ű}{{\H{U}}}1 {ő}{{\H{o}}}1 {Ő}{{\H{O}}}1{ç}{{\c c}}1 {Ç}{{\c C}}1 {ø}{{\o}}1 {å}{{\r a}}1 {Å}{{\r A}}1{€}{{\EUR}}1 {£}{{\pounds}}1}

\lstdefinestyle{stylepython}{
	language=Python,
	basicstyle=\ttfamily,
	commentstyle=\color{red},
	keywordstyle=\color{blue},
	stringstyle=\color{green}, %olive
	numberstyle=\tiny,numbers=left,
	stepnumber=1,numbersep=5pt}

% -------------------------------------------------------
% Information PDF généré
% -------------------------------------------------------
\hypersetup{pdftex, colorlinks=true, linkcolor=blue, citecolor=blue, filecolor=blue, urlcolor=blue, pdftitle=, pdfauthor=Florian Colas, pdfsubject=, pdfkeywords=}

% -------------------------------------------------------
% MACRO
% -------------------------------------------------------
% Pm||Cmax
\newcommand\problemGrahamPm{$P_m||C_{\max}$\xspace}
% P2||Cmax
\newcommand\problemGrahamPII{$P_2||C_{\max}$\xspace}	%apparemment ne supporte pas les chiffres.
% P||Cmax
\newcommand\problemGrahamP{$P||C_{\max}$\xspace}
% Cmax
\newcommand\cmax{$C_{\max}$\xspace}

% -------------------------------------------------------
% Dossier des figures
% -------------------------------------------------------
\graphicspath{{./fig/}}

% -------------------------------------------------------
% Ajout L. Philippe
% Utilisation des Todo inline en macro --> tdi
% -------------------------------------------------------
\usepackage{todonotes}
\newcommand{\tdi}[1]{\todo[inline]{{#1}}{}}
\newcommand{\lp}[1]{\todo[author=LP,color=yellow,inline]{#1}}
\newcommand{\lcc}[1]{\todo[author=LCC,color=green,inline]{#1}}
\newcommand{\fco}[1]{\todo[author=FCO,color=teal,inline]{#1}}
\newcommand{\jb}[1]{\todo[author=JB,color=orange,inline]{#1}}

% -------------------------------------------------------
% Information générales (en attendant la page de garde)
% Utilisé par \maketitle
% -------------------------------------------------------
\title{Évaluation d'algorithmes d'ordonnancement}
\author{Florian Colas}
\date{\today}

% ######################################################################
%
% 				DOCUMENT
%
% ######################################################################
\begin{document}

% #######################################################
% Numérotation des pages
% #######################################################
\pagenumbering{roman}

% =======================================================
% Page de garde
% Utilise Information générales
% Ecrit le titre + auteur + date
% =======================================================
\maketitle

\section*{Dédicaces} \label{sec:dedicace}
je dédie....

\section*{Remerciements} \label{sec:remerciements}
je remercie....

% -------------------------------------------------------
% Table des matières
%
% On renomme en Sommaire (document français)
%
% On définit la profondeur de la table des matières
% -1 partie,    0 Chapitre,
% 1 Section,    2 sous sections,  3 sous sous section
% 4 Paragraphe, 5 Sous paragraphe
%
% Les sections sont numérotées 1 2 3
% -------------------------------------------------------
\renewcommand{\thesection}{\arabic{section}}
\renewcommand{\contentsname}{Sommaire}
\setcounter{tocdepth}{3}	% avant 2 pour la table des matières
\setcounter{secnumdepth}{3}	% pour les section sous et sous sous et paragraphes
\tableofcontents

% =======================================================
% ABSTRACT/RESUME
% =======================================================
\bigskip

%\abstractname{}

\fco{Voir comment regrouper Abstract et Résumé sur une même page, sans utiliser de figure ...}

\selectlanguage{english}
\begin{abstract}
.... when the brief is completed ...
\end{abstract}

\selectlanguage{french}
\begin{abstract}
....une fois le mémoire terminé ...
\end{abstract}

\bigskip

% #######################################################
% Numérotation des pages
% #######################################################
\pagenumbering{arabic}

% =======================================================
% 1 INTRODUCTION GENERALE
% =======================================================
\section{Introduction générale} \label{sec:introductionGenerale}

% -------------------------------------------------------
%ésentation et contexte 
% -------------------------------------------------------
Considérons le problème \problemGrahamP. tel que définit 
  dans la classification à trois champs de 
  Graham \emph{et al.} \cite{graham1979optimization}. 
Ce problème d'ordonnancement consiste à planifier 
  un ensemble $J=\{j_1, j_2, \ldots, j_n\}$ de $n$ jobs, (ou tâches) indépendants, 
  sur $m$ machines identiques (noté $P$) 
  dans le but d'optimiser le temps total de traitement appelé makespan 
  (noté $C_{max}$).
Le temps nécessaire de réalisation de chaque job $p_i~(avec~1 \leq i \leq n)$ 
  est connu à l'avance. 
Un job commencé est complété sans interruption (non préemptif), 
  et est exécuté par une seule machine. 
Une machine ne traite qu'un seul job à la fois. 

Cette question d’ordonnancement est omniprésente, dans la vie 
  pratique, l'industrie, le transport, les institutions, $\ldots$ , 
  mais aussi dans l'allocation des tâches à réaliser par des 
  machines à structure parallèle. 
Le parallélisme est un type d'architecture informatique, dans lequel 
  plusieurs processeurs exécutent, ou traitent
  une application, un calcul simultanément. 
Il aide à effectuer de grands calculs en divisant la charge de travail 
  entre plusieurs processeurs, capables de communiquer et de coopérer. 
Pour ce dernier cas, il est primordial que le calcul de l'attribution 
  de chaque tâche d'un ensemble de jobs aux ressources disponibles, 
  soit effectué le plus rapidement possible, 
  et permette un temps total de traitement de cet ensemble 
  le plus court possible.

\bigskip
% -------------------------------------------------------
% problématique générale 
% -------------------------------------------------------
Mais, comme l'ont démontré Garey et Johnson, 
  \problemGrahamPII est un problème NP-Difficile \cite{garey1978strong}, et 
  \problemGrahamP avec $m \geq 3$ est un problème NP-Difficile 
  au sens fort \cite{garey1982computers}. 
Cependant, \problemGrahamP devient un problème NP-Difficile, 
  du moment que le nombre de machines est fixé \cite{chen1999potts}, 
  comme l'a montré Rothkopf \cite{rothkopf1966scheduling}, 
  qui a présenté un algorithme de programmation dynamique.

Donner la solution optimale à un problème d'ordonnancement 
  (dans notre cas \problemGrahamP) n'est pas réaliste. 
  la résolution de celui-ci demanderait un temps excessif et donc rédhibitoire.
Comme les machines sont identiques, et travaillent à la même vitesse,
la difficulté repose uniquement sur le regroupement des jobs.
La résolution du problème d'ordonnancement va reposer sur des méthodes
  d'approche, qui consistent à calculer en temps polynomial,
  une solution ``assez'' proche de la valeur optimale.

Les jobs sont exécutés sans interruption ni coupure. Donc,
  le makespan ne  peut pas être inférieur à la taille du jobs
  le plus long i.e. $(\max_i\{p_i\})$.
  Dans le cas d'un nombre $n$ de jobs supérieur au nombre $m$ de machines,
  le makespan ne peut pas être inférieur à la moyenne
  des tailles des jobs par machine
  i.e. $\frac{1}{m} \sum_{i=1}^{n} p_i$.
Donc Toutes les solutions ont une borne minimale
  \cite{mcnaughton1959scheduling}: \\

  \begin{center}
  $borne_{min} = \max \{ \max_i\{p_i\}, \frac{1}{m} \sum_{i=1}^{n} p_i \}$
  \label{borneMini}
  \end{center}

L'existence d'une solution qui résout le problème de manière optimale
  n'est pas pensable, à moins que $P = NP$.
Dans la littérature, l'étude d'ordonnancement est très riche et abondante. 
Le but étant d'améliorer le temps de calcul, et d'approcher le résultat optimal. 
Les solutions proposée sont des Heuristiques, ou des approximations.

Le but d'une heuristique (du grec \emph{heuriskein}: trouver) est 
  de trouver une solution respectant les contraintes du problème et 
  de bonne qualité selon le critère d'optimisation considéré. 
  La solution ne sera pas forcément optimale, 
  mais une heuristique efficace tente de trouver une solution de bonne qualité, 
  suivant le temps de résolution imparti.
Ces algorithmes calculent des solutions, dont la borne maximale, 
  au pire des cas n'est pas maîtrisée: une étude du comportement est nécessaire 
  pour définir cette borne maximale.  
  Pour chaque étude de comportement, les notions suivantes sont utilisées :

\begin{itemize}
% RESULTAT DE L'ALGORITHME A
\item $C_m^A(J)$ Le résultat (makespan) de l'ordonnancement
	d'un ensemble $J$ de jobs,
	sur $m$ machines parallèles, identiques,
	obtenu par l'heuristique $A$.
% RESULTAT OPTIMAL
\item $C_m^\star(J)$ Le makespan optimal, idéal, de l'ordonnancement
	d'un ensemble $J$ de jobs,
	sur $m$ machines parallèles identiques.
% RATIO OBTENU/OPTIMAL
\item $\Gamma(A)=\frac{C_m^A(J)}{C_m^\star(J)}$
	Le ratio d'approximation atteint par l'algorithme $A$ au pire cas.
\end{itemize}
   
  Parfois, cette borne n'existe pas. Une simulation permet d'obtenir des 
  informations générales sur les tendances.

\bigskip
% -------------------------------------------------------
% génese du rapport
% -------------------------------------------------------
Parmi les heuristiques les plus étudiées, nous pouvons citer 
  LPT (Longest Time Processing) \cite{graham1966bounds}, 
  LDM (Largest Differencing Method) \cite{karmarkar1982differencing}, 
  COMBINE \cite{lee1988multiprocessor}, et 
  SLACK \cite{della2020longest}.
Ce dernier revisite l'heuristique LPT en appliquant aux données d'entrée du problème, 
  une stratégie gloutonne. 
SLACK devient très compétitif par rapport au autres heuristiques, 
  tant en complexité en temps 
  qu'en ratio d'approximation. 
Cette conclusion est le résultat d'un protocole expérimental 
  basé sur la génération d'instances selon des 
  distributions uniformes et non uniformes.  

Le but de ce mémoire est multiple :
\begin{itemize}
\item reproduire le protocole expérimental de Della Croce \emph{et al.} 
  en implémentant les heuristiques considérées, 
  à l'aide des distribution uniformes et non-uniformes.
\item Élargir ce protocole expérimental à 
  d'autres types de générations aléatoires 
  de listes de jobs (distributions gamma, beta, exponentielles), voire 
  à de vraies log de workloads, 
  afin de soumettre ces heuristiques à un spectre plus large en 
  hétérogénéité et ainsi étudier leur comportement.    
\item tenter d'expliquer pourquoi SLACK est 
  compétitif sur une distribution non-uniforme. 
\item Q et R ?????????
\end{itemize} 

\bigskip
% -------------------------------------------------------
%PLAN
% -------------------------------------------------------
\fco{...Plan une fois terminé...}
État de l'art\\
Mise en place de l'expérience\\
protocole de della Croce et comparaison des résultats\\
Comparaison avec d'autres distributions\\
Résultats\\
Distribution non-uniforme\\
Conclusion \\
Annexes (implémentation de l'expérience)

% =======================================================
% 2 Etat de l'art
% =======================================================
\section{Etat de l'art} \label{sec:etatDeLArt}

\subsection{Introduction} \label{ssec:etatDeLArtIntroduction}

Le problème d'ordonnancement se pose depuis l'apparition des premières machines parallèles, et 
  est d'autant plus d'actualité que les postes personnels sont équipés depuis quelques années, 
  de processeurs (CPU) et de cartes graphiques (GPU) multi-c{\oe}urs.
Les centres de calculs sont dotés de parcs assez uniformes, et maintenant, 
  les clouds, offrent des instances VM qui permettent des environnements 
  d'exécution homogènes.     
Il fait partie de la catégorie des problèmes d'optimisation combinatoire. 
C'est un champ de la recherche opérationnelle, actif depuis plus d'un siècle, 
  et abondant dans la littérature. 

De nombreuses pistes sont explorées. Il est pratiquement impossible de les énumérer toutes, 
  aussi sont présentées ici les plus étudiées, 
  et/ou utilisées pour être comparées ou servir de référent.
  
Nous abordons dans l'ordre, certaines Heuristiques. un type d'algorithme d'approximation, et une résolution du problème basée sur la programmation linéaire.

\subsection{Heuristiques}\label{ssec:Heuristiques}

Les heuristiques représentent la plus grande partie des recherches concernant l'ordonnancement. 
Celles présentées sont basées 
  sur le principe des LS (Liste Scheduling), 
  sur une stratégie gloutonne,   
  sur le le problème bin-packing et  
  sur le problème de partitionnement de nombres. 

\bigskip   
\paragraph{LS (List Scheduling)}
l'idée d'une LS est de stocker l'ensemble des jobs dans celle-ci, les trier dans une ordre particulier et/ou les regrouper selon une règle définie, dans le but de leur assigner une priorité. Ces job sont ensuite affectés un à un à une machine suivant un principe déterminé.

LPT Rule (Longest Time Processing) % (algorithme \ref{algo:LPT})
  proposé par Graham \emph{et al.} \cite{graham1966bounds} 
  est un des algorithmes le plus repris dans la littérature.
(algorithme \ref{algo:LPT})

% -------------------------------------------------------
% LPT 
% -------------------------------------------------------
\begin{algorithm}%[H]
\DontPrintSemicolon
\KwData{instance de \problemGrahamP, avec $m$ machines, $n$ jobs et leur temps d'exécution}

\BlankLine % Petit espace
Trier les jobs de l'ensemble $J$ dans l'ordre décroissant de leur temps
d'exécution et ré-indexer l'ensemble de telle manière à obtenir:
$p_1 \geq p_2 \geq \ldots \geq p_n$

\BlankLine % Petit espace
Parcourir la liste et affecter chaque job à la machine la moins
chargée à ce moment là.
% }
\caption{LPT Rule}
\label{algo:LPT}
\end{algorithm}
% -------------------------------------------------------
% /LPT 
% -------------------------------------------------------


\bigskip   
\paragraph{Stratégie gloutonne}
Della Croce \emph{et al.} \cite{della2020longest} revisitent LPT rule dans le but de l'optimiser. 
L'étude est articulée autour du lien qui existe entre le nombre de machines $m$, le nombre de jobs $n$, 
  la relation qu'il peut y avoir entre les deux, 
  et la probabilité que LPT donne un résultat au pire cas. 
S'ensuit la stratégie gloutonne qui consiste à préparer la LS, avant d’appliquer LPT, comme suit :
\begin{itemize}
\item Trier les jobs par ordre décroissant de leur taille;
\item Découper l'ensemble trié en tuples de m jobs;
\item Soit ``slack'', La différence entre la taille du premier job et la taille du dernier job de chaque tuple;
\item Trier l'ensemble des tuples dans l'ordre décroissant de leur ``slack''.  
\end{itemize} 

\bigskip
(algorithme \ref{algo:SLACK})

% -------------------------------------------------------
% SLACK
% -------------------------------------------------------
\bigskip

% Algorithme Slack
% ---------------------
\begin{algorithm}
\DontPrintSemicolon
\KwData{instance de \problemGrahamP, avec $m$ machines, $n$ jobs et leur temps d'exécution}

%Etape 1
trier la liste des jobs dans l'ordre décroissant des temps nécessaires de traitements \;
%ETAPE 2
réindexer les jobs, de manière à obtenir $p_1 \geq p_2 \geq ... \geq p_n$ \;
%ETAPE 3
découper l'ensemble obtenu en $\lceil \frac{n}{m} \rceil$ tuples de $m$ jobs (ajout
de jobs ``dummy'' de taille nulle pour le dernier tuple, si $n$ n'est
pas un multiple de $m$) \;
%ETAPE 4
considérer chaque tuple avec la différence de temps (SLACK) entre le
premier job du tuple et le dernier.
\begin{align*}
\{ &\{1, ..., m\} &\{m+1,..., 2 \cdot m\} &... \} \\
   &p_1 - p_m     &p_{m+1}-p_{2 \cdot m}  &...
\end{align*} \;
%STEP 5
trier les tuples par ordre décroissant de ``Slack'' et ainsi former un nouvel ensemble
\tcp{e.g: $\{ \{m+1,..., 2 \cdot m\} \{1, ..., m\}\}$ si $p_{m+1} - p_{2 \cdot m} > p_1 - p_m$.}
%STEP 6
appliquer l'ordonnancement (affectation à la machine la moins chargée à
ce moment là) à l'ensemble ainsi obtenu.
\caption{SLACK\label{algo:SLACK}}
\end{algorithm}
% -------------------------------------------------------
% /SLACK
% -------------------------------------------------------

\paragraph{Bin-Packing}
Un des problèmes semblable à \problemGrahamP, est celui de Bin-Packing. Il consiste à 
  ranger des objets, de taille différentes, 
  dans des bacs identiques, tout en minimisant leur nombre.
Ainsi l'ensemble $J=\{j_1, j_2, \ldots, j_n\}$ de $n$ jobs et 
  l'ensemble de leur temps $p_i~(avec~1 \leq i \leq n)$ 
  peuvent être vus respectivement comme
  l'ensemble des objets à ranger et l'ensemble des tailles de ces objets . 
  
% -------------------------------------------------------
% FFD 
% -------------------------------------------------------
\begin{algorithm}%[H]
\DontPrintSemicolon
\KwData{instance de \problemGrahamP, avec $m$ machines, $n$ jobs et leur temps d'exécution}

\BlankLine % Petit espace
Trier les jobs de l'ensemble $J$ dans l'ordre décroissant de leur temps
d'exécution et ré-indexer l'ensemble de telle manière à obtenir:
$p_1 \geq p_2 \geq \ldots \geq p_n$

\BlankLine % Petit espace
Parcourir la liste et affecter chaque job à la machine la moins
chargée à ce moment là.
% }
\caption{FFD}
\label{algo:FFD}
\end{algorithm}
% -------------------------------------------------------
% /FFD 
% -------------------------------------------------------
  
  
  
% -------------------------------------------------------
% MULTIFIT 
% -------------------------------------------------------
\begin{algorithm}%[H]
\DontPrintSemicolon
\KwData{instance de \problemGrahamP, avec $m$ machines, $n$ jobs et leur temps d'exécution}

\BlankLine % Petit espace
Trier les jobs de l'ensemble $J$ dans l'ordre décroissant de leur temps
d'exécution et ré-indexer l'ensemble de telle manière à obtenir:
$p_1 \geq p_2 \geq \ldots \geq p_n$

\BlankLine % Petit espace
Parcourir la liste et affecter chaque job à la machine la moins
chargée à ce moment là.
% }
\caption{MULTIFIT}
\label{algo:MULTIFIT}
\end{algorithm}
% -------------------------------------------------------
% /MULTIFIT  
% -------------------------------------------------------
  
% -------------------------------------------------------
% COMBINE
% -------------------------------------------------------
\begin{algorithm}%[H]
\DontPrintSemicolon
\KwData{instance de \problemGrahamP, avec $m$ machines, $n$ jobs et leur temps d'exécution}

\BlankLine % Petit espace
Trier les jobs de l'ensemble $J$ dans l'ordre décroissant de leur temps
d'exécution et ré-indexer l'ensemble de telle manière à obtenir:
$p_1 \geq p_2 \geq \ldots \geq p_n$

\BlankLine % Petit espace
Parcourir la liste et affecter chaque job à la machine la moins
chargée à ce moment là.
% }
\caption{MULTIFIT}
\label{algo:MULTIFIT}
\end{algorithm}
% -------------------------------------------------------
% /COMBINE  
% -------------------------------------------------------
  
  
  
% -------------------------------------------------------
% LDM 
% -------------------------------------------------------
\begin{algorithm}%[H]
\DontPrintSemicolon
\KwData{instance de \problemGrahamP, avec $m$ machines, $n$ jobs et leur temps d'exécution}

\BlankLine % Petit espace
Trier les jobs de l'ensemble $J$ dans l'ordre décroissant de leur temps
d'exécution et ré-indexer l'ensemble de telle manière à obtenir:
$p_1 \geq p_2 \geq \ldots \geq p_n$

\BlankLine % Petit espace
Parcourir la liste et affecter chaque job à la machine la moins
chargée à ce moment là.
% }
\caption{LDM}
\label{algo:LDM}
\end{algorithm}
% -------------------------------------------------------
% /LDM 
% -------------------------------------------------------


\subsection{Approximation}\label{ssec:Approximation}

Un algorithme d'approximation est un algorithme .....

\paragraph{PTAS (Polynomial Time Approximation Scheme)}
ptas dual 

........

\subsection{Programmation linéaire}\label{ssec:programmationLineaire}

Le problème \problemGrahamP s'inscrit parfaitement dans l'énoncé d'un problème de programmation linéaire.

...

\subsection{conclusion}\label{ssec:etatDeLArtConclusion}

comparaison des algos....

% -------------------------------------------------------
% TABLEAU des heuristiques étudiées dans le rapport
% -------------------------------------------------------
\begin{table}[h] % !! pour éviter qu'il traine au milieu du sommaire !!
\centering
\begin{tabular}{p{3cm} p{3cm} p{1cm} p{4cm} p{4cm}}
% --------------------------
% TITRES
% --------------------------
\hline
algorithme 	& Domaine 
			& [ref] 
			& complexité en temps 
			& ratio d'approximation\\
\hline

% LPT
LPT Rule 	&  
			& \cite{graham1966bounds} 
			& $n \cdot log(n) + n \cdot log(m)$ 
			& $\dfrac{4}{3}$ \\

% SLACK
SLACK   	& 
			& \cite{della2020longest} 
			&$n \cdot log(n) + n \cdot log(m)$ 
			& $\dfrac{4}{3}$ \\

% LDM
LDM   		&  partitionnement 
			& \cite{karmarkar1982differencing} 
			& $n \cdot log(n) + n \cdot log(m)$ 
			& $\dfrac{4}{3}$ \\

% COMBINE
COMBINE 	& bin-packing 
			& \cite{lee1988multiprocessor} 
			& $n \cdot log(n) + n \cdot log(m)$ 
			& ratio \\
%---------------------------
\hline
\end{tabular}
\caption{Heuristiques étudiées}
\label{table:Heuritiques}
\end{table}


\section{Synoptique et mise en place expérimentale}\label{sec:Synoptique}

L’expérience consiste à explorer et comparer le comportement de plusieurs heuristiques. 
C'est un programme qui génère des listes de temps de tâches, 
  exécute ces heuristiques puis récupère, pour chacune d'elle, 
  la simulation de l’ordonnancement calculé dans une liste de m processeurs, 
  le makespan correspondant, ainsi que le temps qu'il a nécessité pour obtenir ce résultat.

Après une vue d'ensemble nous allons voir la génération des listes de tâches (instances), l'implémentation des heuristiques (algorithmes), et comment ces listes et heuristiques sont utilisées et exécutées (campagne) afin d'obtenir un résultat global.

\subsection{Objectifs}\label{ssec:Objectifs}
Generer des instances
Caratériser ces instances
Comparer les cmax et temps d'exec des algos
Faire un lien entre comportement et variables
 N
 M
 Distribution
 Hétérogénéité
 ....
 

\subsection{Technologies utilisées}\label{ssec:TechnologiesUtilisées}
Raison du choix des technos
 Facilité
 Universalité (langage largement utilisé et connu contrairement à ...
 Interprété (pas compilé), donc + portable + évolutif (sources à disposition)
 Abondance de tutoriels et documentations
Inconvénients
 Interprété (trop haut niveau pour les temps d'exec)
 ... 

\begin{itemize}

\item Python
\item Pandas
\item R
\item RStudio
\item ggPlots
\item Github


\end{itemize}

\subsection{Choix des heuristiques}\label{ssec:choixHeuristiques}


\section{Instances} \label{sec:instances}
2 façons différentes de créer un environnement de tests Soit générer les instances, soit les récupérer.

\subsection{génération des instances}\label{ssec:génération des instances}
petite explication de chaque distribution, et graphe (e.g Lambda). 
\fco {trouver les références sur les distributions}

Distribution statistiques différentes
 Uniforme
 Lambda
 Beta
 Exponentielle
 
Utilisée par della Croce
 Non uniforme 
 
 
\subsection{récupération des instances}\label{ssec:récupérationInstances}

\fco {Lire chap 2 et 9. + référence}
Parallel Workload Archive


\subsection{maîtrise de la solution optimale}\label{ssec:maîtriseSolutionOptimale}
complétion à m-1 machines --> lissage de la moyenne par machine --> donc optimal
Dans ce cas, comparer les algorithmes avec le même nombre de machines !

\fco {faire la preuve ou trouver les références ?}

\subsection{indicateurs statistiques et caratéristiques}\label{ssec:indicateursStatistiquesCaratéristiques}


\section{Implémentation des algorithmes}
\label{sec:ImplémentationAlgorithmes}
Tous les algorithmes (étudiés) sont implémentés dans algorithms.py. Les procédures, objets et méthodes directement liés à ceux-ci sont placés dans ScheduleManagment.py i.e ceux qui n'ont pas le même prototype (FFD().). 
Ces algorithmes sont structurés sur le même principe :
\begin{itemize}

\item Les paramètres en entrée :
	\begin{description}

	\item Une liste de coûts : soit l'ensemble des $p_i$. 
	sous la forme $[p_1, p_2, ... , p_n]$. 
	Cette liste n'est pas forcément triée car cela fait partie 
	de l'heuristique, et ce tri peut changer d'une stratégie à l'autre.
	
	\item un nombre de machines m.
	\end{description}

\item Mesure du temps d'exécution. L'heure de début, et l'heure de fin 
sont consignées dans des variables "before" et "after". 
le temps d'exécution est donc mesuré en minutes-secondes.

\item Calcul du temps théorique, en fonction de la complexité théorique 
en temps de l’algorithme, ainsi que de la taille de la liste des temps, 
et du nombre de machines.

\item Déroulement de l'heuristique.

\item Retour d'un object "PSched" contenant les informations suivantes :
	\begin{description}
	\item Le nom de l'algorithme.
	\item Le temps attendu.
	\item le Makespan calculé
	\item Le temps qui a été nécessaire
	\item L'ordonnancement, sous forme de liste.
	\end{description}
\end{itemize}



\subsection{COMBINE}\label{ssec:COMBINE}

les problèmes de COMBINE 
l'algo approximatif (Do while et do until a son importance
Cu - Cl >= alpha A pas forcément ...
Condition d'arrêt n'integre pas le nombre de bin utilisées

\fco {Discuter avec Laurent/Louis-Claude/Julien de ce bug-si connu ou viens de moi}
!!! La borne supérieure  est donnée par LPT rule !!!
puis FFD est exécuté avec, c= borne supérieure + borne inférieure)/2. Or, si FFD est moins performant que LPT, le nombre de bin ne sera jamais égal au nombre de machines considéré. e.g si LPT( [p1,p2,...,pn]) sur m machines donne U, et FFD(U) donne V (avec V > U) FFD entre U et L (L<U) donnera au mieux V.....

COMBINE avec borne sup = 1.1Cmax(LPT), c'est ok

Corrections apportées
algorithme COMBINE
algorithme FFD

\subsection{fonctions annexes}\label{ssec:fonctionsAnnexes}

Quelques mots sur les fonctions appelées par les algorithmes principaux et qui ont leurs importance.

\begin{itemize}
\item FFD \cite{rieck2010basic} (implémentation)
\item Partitions LDM
\end{itemize}

\subsection{Ajout d'un algorithme}\label{ssec:AjoutAlgorithme}
Les sources sont disponibles, et l'application est prévue pour accueillir de nouveaux algorithme, si les règles suivantes sont respectées.

exemple avec MULTIFIT

\section{Campagnes}\label{sec:Campagnes}

\subsection{Génération d'une campagne}\label{ssec:generationCampagne}
Par choix de paramètres
Par exécution pré remplie
Par lecture d'une campagnes

\subsection{Gestion des "seeds"}\label{ssec:gestionSeeds}
si doit reproduire les mêmes instances


\subsection{Structure du fichier résultat}\label{ssec:StructureFichierResultat}

passage en revue des colonnes de de leur provenance


\section{Comparaison des résultats} \label{sec:comparaisonResultats}

Comparer des algo quand on ne connaît pas l'optimal
par rapport au meilleurs résultat

\subsection{Normalisation du MakeSpan}
\label{ssec:normalisationMakeSpan}

trouver la bonne normalisation qui ne biaise pas trop la comparaison.
Mackespan sous forme brute, voire normalisé avec Cmax-optiml donne une courbe  accidentée et pas très lisible

\fco {Discuter avec Laurent/Louis-Claude/Julien si comparaison par graphes, ou voir moindres carrés ou autre méthode}

\begin{itemize}
\item MakeSpan brut
\item MakeSpan - optimal ou meilleur résultat
\item MakeSpan / optimal ou meilleur résultat
\item MakeSpan / Moyenne de tâche par machine
\end{itemize}

\subsection{Temps d'exécution} \label{ssec:tempsExécution}

 Le temps attendu est théorique et non absolu. i.e n log n ne donne pas un temps en minutes secondes.Le temps mesuré lui est donné en minutes secondes --> faire le lien entre estimation théorique et mesuré.
 
D'un algo a l'autre , Python est un langatge de haut niveau, donc pas évident. ce qui peut être comparé, c'est pour un même algo et faire varier N et/ou m. on compare les mêmes instructions....

faire un lien entre le théorique de une valeur pour comparer par la suite les résultats suivants
 
\subsection{Graphes}\label{ssec:graphes}
\fco {Discuter avec Laurent/Louis-Claude/Julien Libre et proposition de graphes, ou conduit}

Différents graphe à disposition
utilisation de ggplot
Modifier l'un d'eux
Abscisse 	= soit n
			= soit m
Couleurs = algorithme
pictogramme = méthode de génération instance (lambda, beta uniforme ...) 
taille = hétérogénéité ?? écart type ?? de l'instance




\section{Résultats} \label{sec:resultats}

\subsection{en fonction de n}\label{ssec:en fonction de n}

\subsection{en fonction de m}\label{ssec:en fonction de m}

\subsection{en fonction de n et m}\label{ssec:en fonction de n et m}

\subsection{en fonction des distributions et hétérogénéité}
\label{ssec:en fonction des distributions et hétérogénéité}



\section{P vers Q et R} \label{sec:P vers Q et R}

je ne sais pas encore



\section{Discussion} \label{sec:discussion}


\section{Conclusion} \label{sec:conclusion}

\section{remerciements}


\bibliographystyle{plain}				% NE PAS ENLEVER !!!!!!!!!!
\bibliography{Bibliographie}			% Utilise Bibliographie.bib


\end{document}
