% ######################################################################
%
% 			PREAMBULE
%
% ######################################################################

% -------------------------------------------------------
% Ce document est un raport. cinquantaine de pages / plusieurs sections
% -------------------------------------------------------
\documentclass[a4paper,12pt]{report}

% -------------------------------------------------------
% Utiliser latin1 pour les accents
% ne pas utiliser utf8 et Latex n'aime pas utf8 ET latin1
%\usepackage[utf8]{inputenc}
% -------------------------------------------------------
\usepackage[utf8]{inputenc}

% Vu dans un forum : utiliser plutôt frenchb, french est obsolete
\usepackage[T1]{fontenc}
\usepackage[english,frenchb]{babel}

\usepackage{amsmath}
\usepackage{amssymb}
\usepackage{amsfonts}
\usepackage{array}
\usepackage{booktabs}
\usepackage{diagbox} % barre oblique pour les cellule à 2 entrées
\usepackage[pdftex]{graphicx}
\usepackage{hhline}
\usepackage{hyperref}
\usepackage{lmodern}
\usepackage{multicol}
\usepackage{subcaption}
\usepackage{supertabular}
\usepackage{textcomp}
\usepackage{xcolor}
\usepackage{xspace}

% -------------------------------------------------------
% Théorèmes
% -------------------------------------------------------
\usepackage{amsthm}
\theoremstyle{plain}				% Choix du style
\newtheorem{theoreme}{Théorème}	% Définition de l'environnement 1
\newtheorem{example}{Exemple}
\theoremstyle{definition}				% Choix du style
\newtheorem{definition}{Définition} %[section]	% Définition de l'environnement définition

% -------------------------------------------------------
% Pour les ALGORITHMES
% linesnumbered	: les lignes sont numérotées
% ruled			: Le caption est en haut et bordé de lignes horizontale
% vlined		: Regroupement des bloc d'instructions par ligne verticales
% -------------------------------------------------------
\usepackage[linesnumbered, ruled, vlined, french]{algorithm2e}

% Commandes en français:
\SetKwInput{KwRes}{R\'esultat}%
\SetKw{WEntree}{\textcolor{red}{Entrée}}
\SetKw{WEntrees}{\textcolor{red}{Entrées}}
\SetKw{WSaisir}{Saisir}
\SetKw{WInitialisation}{\textcolor{red}{Initialisation}}
\SetKw{WTraitement}{\textcolor{red}{Traitement}}
\SetKw{WAssigne}{\textcolor{blue}{prend la valeur}}
\SetKw{WSortie}{\textcolor{red}{Sortie}}
\SetKw{WSorties}{\textcolor{red}{Sorties}}
%
\SetKwIF{Si}{SinonSi}{Sinon}{Si}{alors}{sinon si}{sinon}{fin si}
\SetKwFor{Tq}{Tant que}{faire}{fin tantque}
\SetKwFor{Pour}{Pour}{faire}{fin pour}
\SetKw{WAfficher}{Afficher}
\SetKwRepeat{Repeter}{répéter}{jusqu'à}%
%
\SetKw{Return}{\textcolor{red}{Renvoyer}}%
%
\SetKwProg{Init}{init}{}{}
% mettre les commentaire des algos en bleu

% -------------------------------------------------------
% Pour les ALGORITHMES façon PYTHON
% -------------------------------------------------------
\usepackage{listings}
\lstset{literate={á}{{\'a}}1 {é}{{\'e}}1 {í}{{\'i}}1 {ó}{{\'o}}1 {ú}{{\'u}}1{Á}{{\'A}}1 {É}{{\'E}}1 {Í}{{\'I}}1 {Ó}{{\'O}}1 {Ú}{{\'U}}1{à}{{\`a}}1 {è}{{\`e}}1 {ì}{{\`i}}1 {ò}{{\`o}}1 {ù}{{\`u}}1{À}{{\`A}}1 {È}{{\'E}}1 {Ì}{{\`I}}1 {Ò}{{\`O}}1 {Ù}{{\`U}}1{ä}{{\"a}}1 {ë}{{\"e}}1 {ï}{{\"i}}1 {ö}{{\"o}}1 {ü}{{\"u}}1{Ä}{{\"A}}1 {Ë}{{\"E}}1 {Ï}{{\"I}}1 {Ö}{{\"O}}1 {Ü}{{\"U}}1{â}{{\^a}}1 {ê}{{\^e}}1 {î}{{\^i}}1 {ô}{{\^o}}1 {û}{{\^u}}1{Â}{{\^A}}1 {Ê}{{\^E}}1 {Î}{{\^I}}1 {Ô}{{\^O}}1 {Û}{{\^U}}1{œ}{{\oe}}1 {Œ}{{\OE}}1 {æ}{{\ae}}1 {Æ}{{\AE}}1 {ß}{{\ss}}1{ű}{{\H{u}}}1 {Ű}{{\H{U}}}1 {ő}{{\H{o}}}1 {Ő}{{\H{O}}}1{ç}{{\c c}}1 {Ç}{{\c C}}1 {ø}{{\o}}1 {å}{{\r a}}1 {Å}{{\r A}}1{€}{{\EUR}}1 {£}{{\pounds}}1}

\lstdefinestyle{stylepython}{
	language=Python,
	basicstyle=\ttfamily,
	commentstyle=\color{red},
	keywordstyle=\color{blue},
	stringstyle=\color{green}, %olive
	numberstyle=\tiny,numbers=left,
	stepnumber=1,numbersep=5pt}


% -------------------------------------------------------
% Information PDF généré
% -------------------------------------------------------
\hypersetup{pdftex, colorlinks=true, linkcolor=blue, citecolor=blue, filecolor=blue, urlcolor=blue, pdftitle=, pdfauthor=Florian Colas, pdfsubject=, pdfkeywords=}

% -------------------------------------------------------
% MACRO
% -------------------------------------------------------
% Pm||Cmax
\newcommand\problemGrahamPm{$P_m||C_{\max}$\xspace}
% P2||Cmax
\newcommand\problemGrahamPII{$P_2||C_{\max}$\xspace}	%apparemment ne supporte pas les chiffres.
% P||Cmax
\newcommand\problemGrahamP{$P||C_{\max}$\xspace}
% Cmax
\newcommand\cmax{$C_{\max}$\xspace}

% -------------------------------------------------------
% Dossier des figures
% -------------------------------------------------------
\graphicspath{{./fig/}}

% -------------------------------------------------------
% Ajout L. Philippe
% Utilisation des Todo inline en macro --> tdi
% -------------------------------------------------------
\usepackage{todonotes}
\newcommand{\tdi}[1]{\todo[inline]{{#1}}{}}
\newcommand{\lp}[1]{\todo[author=LP,color=yellow,inline]{#1}}
\newcommand{\lcc}[1]{\todo[author=LCC,color=green,inline]{#1}}
\newcommand{\fco}[1]{\todo[author=FCO,color=blue,inline]{#1}}
\newcommand{\jb}[1]{\todo[author=JB,color=orange,inline]{#1}}

% -------------------------------------------------------
% Information générales (en attendant la page de garde)
% Utilisé par \maketitle
% -------------------------------------------------------
\title{Évaluation d'algorithmes d'ordonnancement}
\author{Florian Colas}
\date{\today}

% ######################################################################
%
% 				DOCUMENT
%
% ######################################################################
\begin{document}

% =======================================================
% Page de garde
% Utilise Information générales
% Ecrit le titre + auteur + date
% =======================================================
\maketitle

% -------------------------------------------------------
% Table des matières
%
% On renomme en Sommaire (document français)
%
% On définit la profondeur de la table des matières
% -1 partie,    0 Chapitre,
% 1 Section,    2 sous sections,  3 sous sous section
% 4 Paragraphe, 5 Sous paragraphe
%
% Les sections sont numérotées 1 2 3
% -------------------------------------------------------
\renewcommand{\thesection}{\arabic{section}}
\renewcommand{\contentsname}{Sommaire}
\setcounter{tocdepth}{3}	% avant 2 pour la table des matières
\setcounter{secnumdepth}{3}	% pour les section sous et sous sous et paragraphes
\tableofcontents

% =======================================================
% Abstract
% =======================================================
\bigskip
\abstractname{}
\bigskip

% =======================================================
% 1 INTRODUCTION
% =======================================================
\section{Introduction} \label{sec:introduction}

\fco{BROUILLON. Trouver une articulation fluide et agréable à lire}

Considérons le problème \problemGrahamP. 
État de l'art du problème d’ordonnancement.
Problème \problemGrahamP. 
n tâches chacune de taille $p_i (avec 1 \leq i \leq n)$, m machines parallèles identiques $( avec 1 \leq j \leq m)$.

puis avec Q et P

Point d'entrée de l'étude : Document de dela Croce

Différentes pistes Heuristiques, BinPacking, Partitionnement de nombre (table \ref{table:Heuritiques})


% -------------------------------------------------------
% TABLEAU des heuristiques étudiées dans le rapport
% -------------------------------------------------------
\begin{table}[h] % !! pour éviter qu'il traine au milieu du sommaire !!
\centering
\begin{tabular}{p{3cm} p{3cm} p{1cm} p{4cm} p{4cm}}
% --------------------------
% TITRES
% --------------------------
\hline
algorithme 	& Domaine 
			& [ref] 
			& complexité en temps 
			& ratio d'approximation\\
\hline

% LPT
LPT Rule 	&  
			& \cite{graham1966bounds} 
			& $n \cdot log(n) + n \cdot log(m)$ 
			& $\dfrac{4}{3}$ \\

% SLACK
SLACK   	& 
			& \cite{della2020longest} 
			&$n \cdot log(n) + n \cdot log(m)$ 
			& $\dfrac{4}{3}$ \\

% LDM
LDM   		&  partitionnement 
			& \cite{karmarkar1982differencing} 
			& $n \cdot log(n) + n \cdot log(m)$ 
			& $\dfrac{4}{3}$ \\

% COMBINE
COMBINE 	& bin-packing 
			& \cite{lee1988multiprocessor} 
			& $n \cdot log(n) + n \cdot log(m)$ 
			& ratio \\
%---------------------------
\hline
\end{tabular}
\caption{Heuristiques étudiées}
\label{table:Heuritiques}
\end{table}

\fco{Enoncer le "problème" auquel on tente de répondre}
Objectif du projet et de ce mémoire :
Apporter un complément expérimental et empirique au document de Della Croce (ce qui manque le plus) et comparer les résultats obtenus
puis dans un deuxième temps, étendre l'expérience aux d'autres problèmes tels que Q et R


% -------------------------------------------------------
% LPT (théorique)
% -------------------------------------------------------

\bigskip
% Algorithme
% ---------------------
\begin{algorithm}[H]
\DontPrintSemicolon
\KwData{instance de \problemGrahamP, avec $m$ machines, $n$ jobs et leur temps d'exécution}

\BlankLine % Petit espace
Trier les jobs de l'ensemble $J$ dans l'ordre décroissant de leur temps
d'exécution et ré-indexer l'ensemble de telle manière à obtenir:
$p_1 \geq p_2 \geq \ldots \geq p_n$

\BlankLine % Petit espace
Parcourir la liste et affecter chaque job à la machine la moins
chargée à ce moment là.
% }
\caption{LPT Rule}
\label{algo:LPT}
\end{algorithm}

% -------------------------------------------------------
% SLACK (théorique)
% -------------------------------------------------------
\bigskip

% Algorithme Slack
% ---------------------
\begin{algorithm}[H]
\DontPrintSemicolon
\KwData{instance de \problemGrahamP, avec $m$ machines, $n$ jobs et leur temps d'exécution}

%Etape 1
trier la liste des jobs dans l'ordre décroissant des temps nécessaires de traitements \;
%ETAPE 2
réindexer les jobs, de manière à obtenir $p_1 \geq p_2 \geq ... \geq p_n$ \;
%ETAPE 3
découper l'ensemble obtenu en $\lceil \frac{n}{m} \rceil$ tuples de $m$ jobs (ajout
de jobs ``dummy'' de taille nulle pour le dernier tuple, si $n$ n'est
pas un multiple de $m$) \;
%ETAPE 4
considérer chaque tuple avec la différence de temps (SLACK) entre le
premier job du tuple et le dernier.
\begin{align*}
\{ &\{1, ..., m\} &\{m+1,..., 2 \cdot m\} &... \} \\
   &p_1 - p_m     &p_{m+1}-p_{2 \cdot m}  &...
\end{align*} \;
%STEP 5
trier les tuples par ordre décroissant de ``Slack'' et ainsi former un nouvel ensemble
\tcp{e.g: $\{ \{m+1,..., 2 \cdot m\} \{1, ..., m\}\}$ si $p_{m+1} - p_{2 \cdot m} > p_1 - p_m$.}
%STEP 6
appliquer l'ordonnancement (affectation à la machine la moins chargée à
ce moment là) à l'ensemble ainsi obtenu.
\caption{SLACK\label{algo:SLACK}}
\end{algorithm}

% -------------------------------------------------------
% LDM (théorique)
% -------------------------------------------------------
\bigskip


% -------------------------------------------------------
% COMBINE (théorique)
% -------------------------------------------------------



/BROUILLON 




\section{Paramètres d"expérience} \label{sec:Paramètres d"expérience}
\fco{mal dit, revoir la phrase}
L’expérience consiste à explorer et comparer le comportement de plusieurs heuristiques (LPT rule, SLACK, LDM et COMBINE). C'est un programme qui génère des listes de n temps de tâches, exécute ces heuristiques puis récupère, pour chacune d'elle, la simulation de l’ordonnancement calculé dans une liste de m processeurs, le makespan correspondant, ainsi que le temps qu'il a nécessité pour obtenir ce résultat.

Après une vue d'ensemble nous allons voir la génération des listes de tâches (instances), l'implémentation des heuristiques (algorithmes), et comment ces listes et heuristiques sont utilisées et exécutées (campagne) afin d'obtenir un résultat global.

\subsection{Synoptique}\label{subsec:Synoptique}



\subsubsection{Objectifs}\label{subsubsec:Objectifs}
Generer des instances
Caratériser ces instances
Comparer les cmax et temps d'exec des algos
Faire un lien entre comportement et variables
 N
 M
 Distribution
 Hétérogénéité
 ....
 

\subsubsection{Technologies utilisées}\label{subsubsec:Technologies  utilisées}
Raison du choix des technos
 Facilité
 Universalité (langage largement utilisé et connu contrairement à ...
 Interprété (pas compilé), donc + portable + évolutif (sources à disposition)
 Abondance de tutoriels et documentations
Inconvénients
 Interprété (trop haut niveau pour les temps d'exec)
 ... 

\begin{itemize}

\item Python
\item Pandas
\item R
\item RStudio
\item ggPlots
\item Github


\end{itemize}

\subsection{Instances} \label{subsec:génération des instances}
2 façons différentes de créer un environnement de tests Soit générer les instances, soit les récupérer.

\subsubsection{génération des instances}\label{subsubsec:génération des instances}
petite explication de chaque distribution, et graphe (e.g Lambda). 
\fco {trouver les références sur les distributions}

Distribution statistiques différentes
 Uniforme
 Lambda
 Beta
 Exponentielle
 
Utilisée par della Croce
 Non uniforme 
 
 
\subsubsection{récupération des instances}\label{subsubsec:récupération des instances}

\fco {Lire chap 2 et 9. + référence}
Parallel Workload Archive


\subsubsection{maîtrise de la solution optimale}\label{subsubsec:maîtrise de la solution optimale}
complétion à m-1 machines --> lissage de la moyenne par machine --> donc optimal
Dans ce cas, comparer les algorithmes avec le même nombre de machines !

\fco {faire la preuve ou trouver les références ?}

\subsubsection{indicateurs statistiques et caratéristiques}\label{subsubsec:indicateurs statistiques et caratéristiques}


\subsection{Implémentation des algorithmes}
\label{subsec:Implémentation des algorithmes}
Tous les algorithmes (étudiés) sont implémentés dans algorithms.py. Les procédures, objets et méthodes directement liés à ceux-ci sont placés dans ScheduleManagment.py i.e ceux qui n'ont pas le même prototype (FFD().). 
Ces algorithmes sont structurés sur le même principe :
\begin{itemize}

\item Les paramètres en entrée :
	\begin{description}

	\item Une liste de coûts : soit l'ensemble des $p_i$. 
	sous la forme $[p_1, p_2, ... , p_n]$. 
	Cette liste n'est pas forcément triée car cela fait partie 
	de l'heuristique, et ce tri peut changer d'une stratégie à l'autre.
	
	\item un nombre de machines m.
	\end{description}

\item Mesure du temps d'exécution. L'heure de début, et l'heure de fin 
sont consignées dans des variables "before" et "after". 
le temps d'exécution est donc mesuré en minutes-secondes.

\item Calcul du temps théorique, en fonction de la complexité théorique 
en temps de l’algorithme, ainsi que de la taille de la liste des temps, 
et du nombre de machines.

\item Déroulement de l'heuristique.

\item Retour d'un object "PSched" contenant les informations suivantes :
	\begin{description}
	\item Le nom de l'algorithme.
	\item Le temps attendu.
	\item le Makespan calculé
	\item Le temps qui a été nécessaire
	\item L'ordonnancement, sous forme de liste.
	\end{description}
\end{itemize}


\subsubsection{LPT rule}\label{subsubsec:LPT rule}


Implémentation Python :

\begin{lstlisting}[style=stylepython]
def lpt(costMatrix, m):
    algoName = "LPT"
    timeExpected = 0.0
    matrixW = costMatrix[:]	
    sched= [] 			
    before = time.time()	
    matrixW.sort(reverse=True)	
    for i in range(len(matrixW)):
        if (len(sched) < m):
            p = sm.Processor()
            p.addJob(matrixW[i])
            sched.append(p)
        else:
            sched.sort(key=attrgetter("jobsTotal"))
            sched[0].addJob(matrixW[i])
    after = time.time()
    makespan = 0.0
    for i in range(len(sched)):
        makespan = max(makespan, sched[i].jobsTotal)
    timeAlgo = after-before
    res = sm.PSched(algoName, timeExpected, makespan, timeAlgo, sched)
    return res

\end{lstlisting}






\subsubsection{SLACK}\label{subsubsec:SLACK}







\subsubsection{LDM}\label{subsubsec:LDM}

reverse ingénierie (pas d'algo dans la litérature)


\subsubsection{COMBINE}\label{subsubsec:COMBINE}

les problèmes de COMBINE 
l'algo approximatif (Do while et do until a son importance
Cu - Cl >= alpha A pas forcément ...
Condition d'arrêt n'integre pas le nombre de bin utilisées

\fco {Discuter avec Laurent/Louis-Claude/Julien de ce bug-si connu ou viens de moi}
!!! La borne supérieure  est donnée par LPT rule !!!
puis FFD est exécuté avec, c= borne supérieure + borne inférieure)/2. Or, si FFD est moins performant que LPT, le nombre de bin ne sera jamais égal au nombre de machines considéré. e.g si LPT( [p1,p2,...,pn]) sur m machines donne U, et FFD(U) donne V (avec V > U) FFD entre U et L (L<U) donnera au mieux V.....

COMBINE avec borne sup = 1.1Cmax(LPT), c'est ok

Corrections apportées
algorithme COMBINE
algorithme FFD

\subsubsection{fonctions annexes}\label{subsubsec:fonctions annexes}

Quelques mots sur les fonctions appelées par les algorithmes principaux et qui ont leurs importance.

\begin{itemize}
\item FFD \cite{rieck2010basic} (implémentation)
\item Partitions LDM
\end{itemize}

\subsubsection{Ajout d'un algorithme}\label{subsubsec:Ajout d'un algorithme}
Les sources sont disponibles, et l'application est prévue pour accueillir de nouveaux algorithme, si les règles suivantes sont respectées.

exemple avec MULTIFIT

\subsection{Campagnes}\label{subsec:Campagnes}

\subsubsection{Génération d'une campagne}\label{subsubsec:Génération d'une campagne}
Par choix de paramètres
Par exécution pré remplie
Par lecture d'une campagnes

\subsubsection{Gestion des "seeds"}\label{subsubsec:Gestion des "seeds"}
si doit reproduire les mêmes instances


\subsubsection{Structure du fichier résultat}\label{subsubsec:Structure du fichier résultat}

passage en revue des colonnes de de leur provenance


\subsection{Comparaison des résultats} \label{subsubsec:Comparaison des résultats}

Comparer des algo quand on ne connaît pas l'optimal
par rapport au meilleurs résultat

\subsubsection{Normalisation du MakeSpan}
\label{subsubsec:Normalisation du MakeSpan}

trouver la bonne normalisation qui ne biaise pas trop la comparaison.
Mackespan sous forme brute, voire normalisé avec Cmax-optiml donne une courbe  accidentée et pas très lisible

\fco {Discuter avec Laurent/Louis-Claude/Julien si comparaison par graphes, ou voir moindres carrés ou autre méthode}

\begin{itemize}
\item MakeSpan brut
\item MakeSpan - optimal ou meilleur résultat
\item MakeSpan / optimal ou meilleur résultat
\item MakeSpan / Moyenne de tâche par machine
\end{itemize}

\subsubsection{Temps d'exécution} \label{subsubsec:Temps d'exécution}

 Le temps attendu est théorique et non absolu. i.e n log n ne donne pas un temps en minutes secondes.Le temps mesuré lui est donné en minutes secondes --> faire le lien entre estimation théorique et mesuré.
 
D'un algo a l'autre , Python est un langatge de haut niveau, donc pas évident. ce qui peut être comparé, c'est pour un même algo et faire varier N et/ou m. on compare les mêmes instructions....

faire un lien entre le théorique de une valeur pour comparer par la suite les résultats suivants
 
\subsubsection{Graphes}\label{subsubsec:Graphes}
\fco {Discuter avec Laurent/Louis-Claude/Julien Libre et proposition de graphes, ou conduit}

Différents graphe à disposition
utilisation de ggplot
Modifier l'un d'eux
Abscisse 	= soit n
			= soit m
Couleurs = algorithme
pictogramme = méthode de génération instance (lambda, beta uniforme ...) 
taille = hétérogénéité ?? écart type ?? de l'instance




\section{Résultats} \label{sec:resultats}

\subsection{en fonction de n}\label{subsec:en fonction de n}

\subsection{en fonction de m}\label{subsec:en fonction de m}

\subsection{en fonction de n et m}\label{subsec:en fonction de n et m}

\subsection{en foonction des distributions et hétérogénéité}
\label{subsec:en foonction des distributions et hétérogénéité}



\section{P vers Q et R} \label{sec:P vers Q et R}

je ne sais pas encore



\section{Discussion} \label{sec:discussion}


\section{Conclusion} \label{sec:conclusion}

\section{remerciements}


\bibliographystyle{plain}				% NE PAS ENLEVER !!!!!!!!!!
\bibliography{Bibliographie}			% Utilise Bibliographie.bib


\end{document}
